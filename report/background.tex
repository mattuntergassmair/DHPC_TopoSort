
\section{Background: Whatever the Background is}\label{sec:background}
%TODO: Johannes

 In this section, we define the topological sort problem and contrast it to BFS and DFS.
 Furthermore, we introduce the basics of the parallel algorithms we use (and a cost analysis?).
 \\
 

\mypar{Topological sorting}
A directed graph can be seen as a binary relation. If the graph is also acyclic it describes a partial order and it is called a directed acyclic graph (DAG). A topological sorting is a total order on a DAG. Let $G = (V,E)$ be a DAG where $V$ is the set of vertices and $E$ is the set of edges. Formally a topological sorting is a function $ord: V \longrightarrow \{1,...,n\}$ where $n = |V|$ such that $\forall (v,w) \in E: ord(v) \leq ord(w)$.
Figure \ref{fig:ts-example} shows a DAG with a corresponding topological sorting: A,I,F,B,E,D,H,G,C. Even though every DAG has at least one topological sorting, mostly there are many different topological sortings for the same graph.

\begin{figure}[!hbp]
 \centering
  \begin{tikzpicture}[->,>=stealth',auto,node distance=1cm,
                    thick,main
                    node/.style={circle,draw,font=\sffamily\scriptsize},text node/.style={draw=none,font=\sffamily\tiny}]

  \node[main node] (1) [draw=black!80,text=black!80] {A};
  \node[main node] (3) [right of=1, node distance=1.2cm]{B};
  \node[main node] (7) [below of=3, node distance=1.2cm] {E};
  \node[main node] (4) [left of=7, node distance=1.2cm] {D};
  \node[main node] (5) [below of=7, node distance=1.2cm] {H};
  \node[main node] (6) [left of=4, node distance=1.2cm] {C};
  \node[main node] (8) [left of=5, node distance=1.2cm] {G};
  \node[main node] (9) [below of=8, node distance=1.2cm] {I};
  \node[main node] (2) [left of=8, node distance=1.2cm] {F};
  


  


  \path[every node/.style={font=\sffamily\small}]
    (1) edge (3)
    (3) edge (7)
    (7) edge (4)
    (4) edge (6)
    (7) edge (5)
    (5) edge (8)
    (8) edge (6)
    (9) edge (5)
    (9) edge (2)
    (2) edge (8)
    ;
\end{tikzpicture}

\caption{Example DAG where A,I,F,B,E,D,H,G,C is one possible topological sorting}
\label{fig:ts-example}
\end{figure}

Topological sortings are often beeing used to schedule tasks based on their dependencies. Every vertice represents a task and the edges represent the relation between the tasks.

Although the results of the breadth-first-search (BFS) algorithm seem to be similar to topological sortings, they are not equivalent. BFS and topolocical sorting algorithms both yield a sequence of nodes, containing each node exactly once. The important difference between the results of BFS and a topological sorting is, that a topological sorting is a total order with respect to the partial order represented by the input graph. But BFS can also return sequences which are not total orders. A simple example is shown in figure \ref{fig:diff-bfs}. The visiting sequence A,C,B is valid in BFS but is not a topological sorting. The only valid topological sorting for the example graph is A,B,C.


\begin{figure}[!hbp]
\centering
 
  \begin{tikzpicture}[->,>=stealth',auto,node distance=1cm,
                    thick,main
                    node/.style={circle,draw,font=\sffamily\scriptsize},text node/.style={draw=none,font=\sffamily\tiny}]

  \node[main node] (1) [draw=black!80,text=black!80] {A};
  \node[main node] (2) [below right of=1, node distance=1.2cm]{B};
  \node[main node] (3) [below of=1, node distance=1.2cm] {C};
  

  \path[every node/.style={font=\sffamily\small}]
    (1) edge (2)
    (2) edge (3)
    (1) edge (3)
    ;
\end{tikzpicture}

\caption{A,C,B is a valid sequence in breadth-first-search (BFS) and depth-first-search (DFS) but not a topological sorting}
\label{fig:diff-bfs}
\end{figure}

A sequential algorithm yielding a topological sorting has been suggested by Kahn \cite{kahn1962topological}. Another algorithm has been published by Tarjan  \cite{tarjan1976edge} and is based on depth-first-search (DFS). Both algorithms have an asymptotic complexity of $\mathcal{O}(|V|+|E|)$. It is to mention that DFS alone does not necessarily return a topological sorting. This can be shown with the same argument as for BFS using figure \ref{fig:diff-bfs}.

 \mypar{Parallel algorithm} 
 1983 M. C. Er \cite{er1983parallel} came up with a parallel approach to retrieve a topological sorting. The algorithm works in 5 steps:
 \begin{enumerate}
        \item Build the graph from the given partial order (optional if the problem is already stated as a graph).
        \item Add a special node value to every node and initialize it to zero
        \item Visit all source nodes (nodes with an indegree of zero) and set their node values to one.
        \item Start from all source nodes in parallel and proceed with all successor nodes as follows: Let $N_p$ be the node value of the current source node and $N_s$ the node value of the currently observed successor node. Then the algorithm checks if $N_s \leq N_p$ and sets the value of the successor to $N_p + 1$, if so. This step is iterated until there are no further successor nodes. If during this step a value higher than the total amount of nodes is beeing assigned to a node, this means that the input graph is not acyclic and the algorithms stops.
        \item List all the nodes in ascending order of node values.
 \end{enumerate}

The correctness of the algorithm relies on another condition. To avoid the situation that two processes or threads try to update a node value at the same time with different values, M. C. Er proposes a synchronization after every step. This results in the fact that if two processes or threads try to update the same node's value, they want to write the same value. This is because they must be at the same iteration step due to the proposed barrier synchronization. This comes at the price of a lower performance. Also with this approach several processes or threads could follow the same paths in the graph because there is no mechanism preventing a process or thread of following a path which already has been processed by another process or thread.

The asymptotic parallel runtime of the above algorithm is described by M. C. Er as $\mathcal{O}(D_{max})$, where $D_{max}$ is defined as the maximum distance between a source node and a sink node. This runtime is hard to achieve in practice if one implements step 5 of the algorithm via sorting the nodes with respect to their values. It is not mentioned by M. C. Er how to create the result list of step 5.


\mypar{Improvements} 
The approach proposed in this report does not use node values. Instead the node is directly put into the solution list. This avoids having to sort the nodes by their value at the end. Also the barriers are not necessary anymore. However, it must be made sure that no node is written more than once to the result list and race conditions while writing to the solution list must be avoided. This approach addresses the problem of several processes following the same path with introducing a parent counter. This counter is a special value of each node, stating how many parent nodes it has. At the beginning this value is set to the actual amount of parents. During the algorithm each process arriving at a node will decrease the counter by one. It will only follow the node if the parent counter is zero. Thus only the last arriving process will follow the path. It has to be taken care of the possible race condition while updating the parent counter.



 
 
 
 \begin{invisible}
 % Serial TS
 \mypar{Topological sorting}
 \begin{itemize}
  \item What is topological sort, difference to BFS
  \item Input: A set of dependencies (aka partial orders) of the form A $\rightarrow$ B ``A must come before B''
  \item Output: A sequence (aka total order) containing all nodes exactly once. All partial orders must be kept.
  \item Solution not unique
  \item Minimal Example: A->B, A->C, B->C. Valid BFS traversal order: A, C, B. Invalid for TS.
  \item TS can (serially) be solved with Kahn's algorithm \cite{kahn1962topological} or DFS and Backpropagation (Tarjan \cite{tarjan1976edge}). % See Wikipedia
        Note that TS is not equivalent to DFS, e.g. for A->B, B->C, A->D, D->E, DFS and Backpropagation yields A, B, C, D, E, but another valid TS is A, B, D, C, E
  \item Asymptotic runtime: O(|V| + |E|)
As an aside, don't talk about "the complexity of the algorithm.'' It's incorrect,
problems have a complexity, not algorithms.  
 \end{itemize}

 % Multithreaded TS
 \mypar{Parallel algorithm}
 \begin{itemize}
  \item Short overview over algorithm of MC Er
  \item Parallelization over child nodes
  \item His idea with barrier in each step such that even if the index is written by multiple threads, they write the same number => Avoid race condition at writing the index
  \item Our idea: Instead of writing an index, directly write to solution list. As a consequence, we have to make sure that node is written to solution only once. And of course there is a race condition on writing to solution list.
  \item Our idea: First, count (in parallel) how many parents each node has. Each time a node is visited, decrement counter and only write to solution if counter is zero. Of course, there is a race condition on the parent counter.
  \item 3 synchronization points (that is, bottlenecks): 1. Barrier after each level, 2. Lock solution list for appending new nodes, 3. Lock parent counter for decrementing it and checking if it is zero.
  \item Cost
 \end{itemize}

 
\end{invisible}
