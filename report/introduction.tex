\section{Introduction}\label{sec:intro}

\mypar{Motivation}
Topological sorting is used to yield a total order from a set of partial orders. For example this is needed for scheduling tasks which depend on other tasks. It can also be used by a linker to resolve software dependencies. Furthermore, a compiler can use topological sorting during static code analysis to rearrange different code slices.

Like other graph algorithms, topological sorting is an example of a heavily memory bound problem. As such, parallelizing the topological sorting algorithm is challenging and relevant, as hardware trends suggest future applications of high performance computing to shift towards the memory bound realm.


\mypar{Related work} 
M.C. Er \cite{er1983parallel} proposed a parallel algorithm for topological sorting in 1983. This graph algorithm assigns every node a value in parallel. To ensure correctness a barrier is necessary after each node processed. The asymptotic runtime of this algorithm is limited by the longest distance between a source and a sink node. Unfortunately it is left unclear how to retrieve a topological sorting from the node values without sorting them. Furthermore it is not considered that nodes can be processed by several processes, which does not break correctness but decreases the performance. Also, there is no information about work balancing, such that the algorithm is not practicable in the proposed shape.

Ma \cite{ma1997efficient} proposed an algorithm solving the problem using an adjacency matrix of the graph and calculating the transitive closure of that matrix. This results in an asymptotic runtime  of $\mathcal{O}(\log^2 |V|)$ on $\mathcal{O}(|V|^3)$ processors, where $V$ is the set of vertices of the graph. This analysis uses the Parallel Random Access Machine (PRAM) model and because of the exponential growth of needed processors the algorithm is not useful in practice.

Both algorithms are proposed without code and have not been implemented by their authors.

In this report we propose a parallel algorithm based on the one by M.C. Er. It addresses the described issues preventing it from being of use in practice and has been implemented and evaluated.


\begin{invisible}
  \begin{itemize}
  \item Software Dependencies
  \item Maybe, to flesh out: Admittedly a bit academic, but interesting problem nevertheless, because memory bound => This is the future of HPC
  \end{itemize}
\end{invisible}


\begin{invisible}
  \begin{itemize}
  \item MC Er Paper \cite{er1983parallel}: Unclear how to retrieve a sorted list from values without sorting and threads might chase other threads. No words about load balancing => Not practicable
  \item Ma Paper \cite{ma1997efficient}: Theoretical analysis in PRAM model, not practicable.
  \item Both cases: No code
  \item Our contribution: (1) Modified algorithm based on MC Er. 1. Sorted list is directly extracted. 2. only one thread continues when multiple threads meet. 3. Ensure load balancing
                          (2) Actual implementation for shared memory architecture
  \end{itemize}
\end{invisible}

