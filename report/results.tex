\section{Experimental Results}\label{sec:exp}
%
In this section, we present our experimental setup and the impact of the different optimizations and implementations described in the previous section.
\par\medskip
%
\begin{invisible}
	Ist wahrscheinlich doch keine Gute Idee, das Abstiming zu plotten - die Schnellen Algorithmen haben fast identische laufzeiten und die anderen sind seeeeeeeeeehr schlecht
\end{invisible}
%
\mypar{Dependency on Graph Type} Before discussing the differences between the algorithms that we implemented, we want to have a closer look at the input graphs.
Figure \ref{fig:strongscaling_graphtypes} shows the strong scaling of the BITSET algorithm for several different input graphs:
The multichain graphs consist of multiple chains of length 100 and 1000 respectively, \ldots and the software graph was constructed accorting to \ldots \\
\begin{invisible}
	{\Large TODO:} quick description of graphs here
\end{invisible}
It should be noted at this point that all of the analyzed graphs are artificial (i.e. they do not come from real-world datasets but are rather constructed to meet some requirements).
The DAGs we used for our analysis are sparse, meaning that the number of edges $|E|$ in the graph is much smaller than the maximum number of vertices $|V|\cdot(|V|-1)$ and in particular that the number of edges in the graph scale linearly with the graph size, $|E| \sim \mathcal{O} ( |V| )$. \\
It can be expected that the following performance analysis based on artificial graphs is a good indicator of how the presented topological sorting algorithms would scale for real-world graphs with similar parameters.
To this point however, the performance of the algorithms on real-world graphs was not tested yet, and an investigation of such could be content of further research.
\par\medskip
%
For the remaining performance analysis, only the random graph with node degree 16 is considered.
Plots for other graph types can be found in the repository of our project which is listed at the end of this report.

% \begin{figure}[ht]
% 	\centering
% 	\includegraphics[width=\columnwidth]{plots/abstiming_gtRANDOMLIN_n1000000_deg16.pdf}
% 	\caption{<+caption text+>}
% 	\label{fig:<+label+>}
% \end{figure}
 
\begin{figure}[ht]
	\centering
	\includegraphics[width=\columnwidth]{plots/strongscaling_gtALL_n1000000.pdf}
	\caption{<+caption text+>}
	\label{fig:strongscaling_graphtypes}
\end{figure}
 
\begin{figure}[ht]
	\centering
	\includegraphics[width=\columnwidth]{plots/strongscaling_gtRANDOMLIN_n1000000_deg16.pdf}
	\caption{<+caption text+>}
	\label{fig:<+label+>}
\end{figure}
 
\begin{figure}[ht]
	\centering
	\includegraphics[width=\columnwidth]{plots/weakscaling_gtRANDOMLIN16_n1000000_deg16.pdf}
	\caption{<+caption text+>}
	\label{fig:<+label+>}
\end{figure}



\mypar{Hardware and compiler}
  \begin{table}[h]
    \centering
    \begin{tabular}{ll}
    \toprule
    Processor        & Intel Xeon E5-2697 (Ivy bridge) \\
    Max. clock rate  & 3.5 GHz (with TurboBoost)\\
    \# Sockets       & 2 \\
    Cores / socket   & 12 \\
    Threads / socket & 24 \\
    LLC / socket     & 30 MB \\
    \midrule
    Compiler and flags & GCC 4.8.2, -O3\\
    \bottomrule
    \end{tabular}
    \caption{Hardware and compiler used for benchmarks}
    \label{tab:hardware}
  \end{table}
 
 The following experiments were run on a 24-core system consisting of 2 Intel Xeon E5 processors (see table \ref{tab:hardware}).
 Hyperthreading was not used for the benchmarks.
 The implementations were written in \Cpp, using OpenMP and GCC atomic built-in functions. The graph was stored in an adjacency list.
\begin{invisible}

 \mypar{Benchmarks}
 For each item, mention graph type, number of nodes, node degree, optimizations from above
 \begin{itemize}
  \item Influence of barrier. Introduce multichain graph, Strong Scaling dynamic nobarrier opt2 MULTICHAIN100 vs bitset global opt1 MULTICHAIN100 Why Multichain100: Should be dominated by barrier%bitset_global is bitset without local solutions, because dynamic_nobarrier has no local solutions
  \item Influence of local solution. Strong Scaling bitset MULTICHAIN10000 vs bitset global MULTICHAIN10000 Why Multichain10000: Should be dominated by pushbacks to solution
  \item Influence of atomic counter check. Strong Scaling RANDOMLIN 100000 opt = true vs opt = false for worksteal and bitset
  \item Strong Scaling software graph \cite{musco2014generative}
  \item Strong Scaling random graph with different degrees
  \item Vertex Scaling plot random graph
 \end{itemize}

\mypar{Results}

Questions to each plot
\begin{itemize}
 \item What is the performance penalty of the barrier?
 \item How well performs the local solution compared to locking the 
 \item How does the atomic counter check scale compared to the locked version?
 \item How does the best version of our code scale on a real-world example?
 \item How does the best version of our code scale in a slightly artificial scenario?
\end{itemize}
\end{invisible}
