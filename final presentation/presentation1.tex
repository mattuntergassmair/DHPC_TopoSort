\input{./beamerheader.sty}

\author[]{Kevin Wallimann \quad Johannes Baum \quad Matthias Untergassmair}

%----------------------------------------------------------------------------------------
%	TITLE PAGE
%----------------------------------------------------------------------------------------

\title[Topological Sorting]{Parallel Topological Sorting} % The short title appears at the bottom of every slide, the full title is only on the title page
\subtitle{Design of High Performance Computing, Fall 2015}

\institute[ETHZ]{ ETH Zürich }
%\date{\today} % Date, can be changed to a custom date
\date{December 14, 2015}
	\begin{document}

\begin{frame}
\titlepage % Print the title page as the first slide
\end{frame}

%----------------------------------------------------------------------------------------
%	PRESENTATION SLIDES
%----------------------------------------------------------------------------------------

%------------------------------------------------
\section{Overview}
%------------------------------------------------


\begin{frame}
\frametitle{Overview}

\begin{itemize}
\item DAG defines partial order
\item Topological sorting defines one total order on a DAG
\item Parallel algorithm: finds one topological sorting of a given DAG
\end{itemize}

\end{frame}

%------------------------------------------------
\section{Difference to BFS}
%------------------------------------------------

\begin{frame}
\frametitle{Difference to BFS}

\begin{itemize}
\item BFS visits every node
\item Topological sorting algorithm needs to visit every edge
\item Thus no BFS optimizations like bottom-up-BFS possible
\end{itemize}

Example:

\begin{figure}[!hbp]
 
  \begin{tikzpicture}[->,>=stealth',auto,node distance=1cm,
                    thick,main
                    node/.style={circle,draw,font=\sffamily\scriptsize},text node/.style={draw=none,font=\sffamily\tiny}]

  \node[main node] (1) [draw=black!80,text=black!80] {A};
  \node[main node] (2) [below right of=1, node distance=1.2cm]{B};
  \node[main node] (3) [below of=1, node distance=1.2cm] {C};
  

  \path[every node/.style={font=\sffamily\small}]
    (1) edge (2)
    (2) edge (3)
    (1) edge (3)
    ;
\end{tikzpicture}

\end{figure}

Consider order A,C,B
$\rightarrow$ valid in BFS, invalid in topological sorting


\end{frame}

%------------------------------------------------
\section{Input graphs}
%------------------------------------------------

\begin{frame}
\frametitle{Random graph}
\begin{figure}[!hbp]
    \includegraphics[height=0.7\textheight]{img/random_lin10.pdf}
\end{figure}

\end{frame}

\begin{frame}
\frametitle{Software dependency graph}
\begin{figure}[!hbp]
    \includegraphics[height=0.7\textheight]{img/software10.pdf}

\end{figure}
{\color{gray}\tiny Musco, V. et al. (2014) "A Generative Model of Software Dependency Graphs to Better Understand Software Evolution."}
\end{frame}


%------------------------------------------------
\section{Implementations}
%------------------------------------------------

\begin{frame}
\frametitle{Implementations}

List based approach:
\begin{itemize}
\item Nodes to be visited ("front") are stored in a linked list
\item Every thread builds its local list
\item Local lists are merged and redistributed at synchronization points to distribute work equally
\end{itemize}

Array based approach:
\begin{itemize}
\item Nodes to be visited are stored in a global bytemap which maps the node id to a boolean value
\end{itemize}

\end{frame}

%------------------------------------------------
\section{Results}
%------------------------------------------------

\begin{frame}
\frametitle{Results - Xeon Phi}



\end{frame}


\end{document} 
